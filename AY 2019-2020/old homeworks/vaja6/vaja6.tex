\documentclass[a4paper,12pt]{article}
\usepackage[utf8]{inputenc}
\usepackage[T1]{fontenc}
\usepackage[slovene]{babel}

\title{Računalniški praktikum II}
\author{Naloga za samostojno reševanje 6}
\date{5. april 2019}

\begin{document}
\maketitle
\thispagestyle{empty}

\noindent
Datoteka \texttt{plot.html} izrisuje graf kvadratne funkcije $f(x) = 0.01 \cdot x^2 - 100$. To počne v zelo slabi ločljivosti, saj definira samo $9$ točk, in sicer na sledeči način:

\begin{verbatim}
<script>
var points = [
    [-200, 300],
    [-150, 125],
    [-100, 0],
    [-50, -75],
    [0, -100],
    [50, -75],
    [100, 0],
    [150, 125],
    [200, 300]
];
</script>
\end{verbatim}

\bigskip\noindent
Izris krivulje izboljšajte po sledečih enostavnih korakih:
\begin{enumerate}
\item Zgornjo definicijo točk nadomestite s programsko kodo PHP, tako da bo funkcija imela ločljivost vsaj 100 točk. Posamezna točka je izračunana kot $[x, \; 0.01 \cdot x - 100]$. Horizontalni razmak med točkami naj bo enakomeren, t.j. števec $x$ naj teče od $x = -200$ do $x = 200$, z enakomernim razmakom $k$.

\item Splošna oblika kvadratne funkcije je
$$
f(x) = ax^2 + bx + c
$$
To pomeni, da so parametri pri zgornji krivulji enaki $a = 0.01$, $b = 0$ in $c = -100$. Omogočite nastaviteh teh treh parametrov ter ločljivosti $k$ preko povezave URL, npr.:
$$
\texttt{plot.php?a=0.02\&b=-10\&c=50\&k=4}
$$
Če kateri od parametrov ni nastavljen, naj prevzame vrednost $a = 0.01$, $b = 0$, $c = -100$ oz. $k = 50$.

\item Spodnjo vrstico
\begin{verbatim}
f(x) = 0.01 &middot; x<sup>2</sup> + 0 &middot; x - 100
\end{verbatim}
dopolnite s kodo PHP, ki bo vrednosti izbranih parametrov $a$, $b$ in $c$ pravilno vstavila v zapis funkcije. Npr. funkcija s parametri $a = 0.02$, $b = -10$, in $c = 50$ bo zapisana kot
$$
f(x) = 0.02 \cdot x^2 - 10 \cdot x + 50
$$

\item Pod graf dodajte HTML obrazec za nastavitev parametrov $a$, $b$, $c$ in $k$, ki je izpolnjen z njihovimi trenutnimi vrednostmi. Spremembe uporabnik potrdi s klikom na gumb `Refresh', ki izriše nov graf funkcije. Vnos realnih števil lahko nastavite z lastnostjo \texttt{step}, npr.:
\begin{verbatim}
<input type="number" name="a"
       min="-10" max="10" step="0.01">
\end{verbatim}

\item Prenos parametrov realizirajte še po metodi POST.
\end{enumerate}

\end{document}